\documentclass{book}

% Core packages
\usepackage{amsmath}
\usepackage{graphicx}
\usepackage[hyphens]{url}
\Urlmuskip=0mu plus 1mu

% caption/subcaption: load caption before subcaption
\usepackage{caption}
\usepackage[labelformat=simple]{subcaption}

\usepackage{multirow}
\usepackage{xcolor}
\usepackage{titlesec}
\usepackage{cite}

\emergencystretch=1em

% hyperref should be loaded last (except packages that explicitly must be after it)
\usepackage[hidelinks]{hyperref}

% small helper command: use one argument
\newcommand{\rmk}[1]{\textcolor{blue}{#1}}

\setcounter{secnumdepth}{4}
\setcounter{tocdepth}{4}

\title{Thesis}
\author{Tom Laclavère}
\date{\today}

\begin{document}

\frontmatter
\maketitle
\clearpage
\thispagestyle{empty}
\tableofcontents

\mainmatter

\part{Cosmological Context}

\chapter{Standard Model of Cosmology}
From Dodelson

\section{The expanding Universe}

\section{The fundamental equations of Cosmology}

\section{History of the Universe}

\section{Problems of the Standard Model}

\chapter{CMB, Inflation, and B-modes}

\section{Inflation theory}

\section{Cosmic Microwave Background}

\section{CMB polarisation}

\section{CMB contamination: astrophysical foregrounds}

\section{CMB past and future observations}

\part{QUBIC instrument: the first bolometric interferometer for cosmology}

\chapter{Bolometric Interferometry and QUBIC}

\section{Classical telescope for CMB physics: Imager}

\section{Innovative approach: Bolometric Interferometer}

\section{QUBIC Collaboration}

\section{Instrumental Design}

\chapter{Bolometric Interferometry to Spectral Imaging}

\section{Synthesized Beam}

\section{Spectral Imaging}

\chapter{GPS for Self-Calibration}
% See GPS chapter.

\part{Map-Making}

\chapter{Forward Map-Making}

\section{Frequency Map-Making}
% See FMM paper.

\section{Component Map-Making}
% See CMM paper.

\section{Realistic TOD Simulation}
% Chapter order to be reviewed.

\subsection{How to define Realistic TOD}

\subsection{Convolution approximation}

\subsection{TOD convergence problem: Pixelisation vs Number of Frequency}

\subsection{\texorpdfstring{Effect of $N_{\text{sub,TOD}} = N_{\text{sub,rec}}$}{Effect of Nsub\_TOD = Nsub\_rec}}

\subsection{External data addition effect (Planck)}

\subsection{Simulation hyperparameters optimisation: $N_{\text{sub}}$, $N_{\text{pointings}}$, $N_{\text{iter}}$, $N_{\text{loop}}$ (CMM), \dots}

\chapter{Inverse Map-Making}

\section{Neural Network Map-Making}
% Wait NNMM paper.

\chapter{Atmospheric Mitigation}
% Wait Atm paper.

\chapter{QUBIC Forecasts}

\section{Studied Cases}

\subsection{CMB without foregrounds: QUBIC raw sensitivity}

\subsection{CMB + Dust (d0, d1, d6, \dots): Simple forecasts}

\subsection{CMB + Dust (d0, d1, d6) + Synchrotron (s0, s1): Realistic forecasts}

\section{Results}

\subsection{Compare QUBIC configurations}

\subsection{Dual bands without spectral imaging: ``Imager case''}

\subsection{Dual bands with spectral imaging: ``BI case''}

\subsection{Ultra-wide band: ``Extreme BI case''}

\subsection{Successive mono-band case: ``Conservative case''}

\section{Compare QUBIC's map-making algorithms}

\subsection{FMM: Cross-spectra Analysis}

\subsection{NN-FMM: Cross-spectra Analysis}

\subsection{CMM: Parametric Analysis}

\subsection{NN-CMM: Parametric Analysis}

\subsection{CMM: Blind Analysis}

\subsection{Addition of external data for component separation}

\subsection{Exploring exotic models}

\backmatter
\begin{thebibliography}{9}
% add bib items here
\end{thebibliography}

\end{document}
