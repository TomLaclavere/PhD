\documentclass[twoside, 12pt]{book}

% Import packages and utils
% Core packages
\usepackage{amsmath}
\usepackage{graphicx}
\usepackage[hyphens]{url}
\Urlmuskip=0mu plus 1mu
\usepackage{subfiles}

% caption/subcaption: load caption before subcaption
\usepackage{caption}
\usepackage[labelformat=simple]{subcaption}

\usepackage{multirow}
\usepackage{xcolor}
\usepackage{titlesec}

\emergencystretch=1em

% hyperref should be loaded last (except packages that explicitly must be after it)
\usepackage[hidelinks]{hyperref}

\usepackage{minitoc}
\usepackage{subfiles}
\usepackage[
    backend=biber,
    style=authoryear,
    refsection=chapter
]{biblatex}

% Horizontal line
\newcommand{\HRule}{\rule{\linewidth}{0.7mm}}
\newcommand{\Hrule}{\rule{\linewidth}{0.3mm}}

% Remark in blue
\newcommand{\rmk}[1]{\textcolor{blue}{#1}}


% Import Figures for each chapter
\graphicspath{{Figures/}{chapters/gps_calsource/Figures/}}

% Import Bibliography for each chapter
\addbibresource{chapters/gps_calsource/gps_calsource.bib}

\begin{document}

\dominitoc

\begin{titlepage}

    \begin{figure}[!t]
        \centering
        \begin{minipage}{0.28\textwidth}
            \centering
            \includegraphics[width=6cm]{Figures/logo_UPC.pdf}
        \end{minipage}
        \hfill
        \begin{minipage}{0.28\textwidth}
            \centering
            \hspace{0.5cm}
            \includegraphics[width=4cm]{Figures/logo_ED.jpg}
        \end{minipage}
        \hfill
        \begin{minipage}{0.28\textwidth}
            \centering
            %\hspace{-3cm}
            \includegraphics[width=4cm]{Figures/logo_APC.png}
        \end{minipage}
    \end{figure}

    \begin{center}

        {\large Université Paris Cité}\\
        \vspace{1cm}
        {\large École doctorale des Sciences de la Terre et de l’Environnement et Physique de l’Univers - ED560}\\
        \vspace{1cm}
        %\vfill

        {\large Laboratoire AstroParticule et Cosmologie (APC) - Groupe Cosmologie}

        \vspace{0.5cm}
        \HRule \\[0.1cm]
        { \Large \bfseries QUBIC Data Analysis: Realistic astrophysical components reconstruction and atmospheric mitigation using spectral imaging}
        \Hrule \\

    \end{center}

    %\vfill

    \begin{center}
        \vspace{0.5cm}
        Par \textsc{\Large Tom Laclavère}\\[0.1cm]
        \vspace{0.5cm}
        Thèse de doctorat de Physique de l’Univers\\[0.4cm]

        \vspace{0.15cm}

        Dirigée par Jean-Christophe Hamilton \\[0.1cm]
        Et co-encadrée par Pierre Chanial
    \end{center}

    \vspace{0.1cm}


    \begin{center}
        \textit{Soutenue publiquement le ... devant un jury composé de :}
        \vspace{1cm}
        \\
        \begin{tabular}{lll}
            \\
            \textsc{Jean-Christophe HAMILTON} & \textsc{DR, Université Paris Cité} & Directeur de thèse \\
        \end{tabular}\\[1cm]
    \end{center}

    \newpage
\end{titlepage}

\newpage

\hspace{5cm} \textbf{\Huge Résumé}

\vspace{1cm}

\vspace{0.5cm}

\textbf{Mots clés :} Fond diffus cosmologique - Inflation - Interférometrie Bolométrique - Imagerie Spectrale - Séparation de composantes - Map-Making

\newpage

\hspace{5cm} \textbf{\Huge Abstract}

\vspace{1cm}

\vspace{0.5cm}

\textbf{Keywords :} Cosmic Microwave Background - Inflation - Bolometric Interferometry - Spectral-Imaging - Component Separation - Map-Making

\newpage

\newpage
\thispagestyle{empty}
\mbox{}
\newpage

\hspace{4cm} \textbf{\Huge Remerciements}

\vspace{1cm}

\vspace{2cm}

\newpage

\tableofcontents
\addstarredchapter{Table of contents}

% \addstarredchapter{List of figures}
% \listoffigures

% \addstarredchapter{List of tables}
% \listoftables

% \input{chapters/intro.tex}
% \addstarredchapter{Introduction}


\part{Cosmological Context}

\chapter{Standard Model of Cosmology}
From Dodelson

\section{The expanding Universe}

\section{The fundamental equations of Cosmology}

\section{History of the Universe}

\section{Problems of the Standard Model}

\chapter{CMB, Inflation, and B-modes}

\section{Inflation theory}

\section{Cosmic Microwave Background}

\section{CMB polarisation}

\section{CMB contamination: astrophysical foregrounds}

\section{CMB past and future observations}

\part{QUBIC instrument: the first bolometric interferometer for cosmology}

\chapter{Bolometric Interferometry and QUBIC}

\section{Classical telescope for CMB physics: Imager}

\section{Innovative approach: Bolometric Interferometer}

\section{QUBIC Collaboration}

\section{Instrumental Design}

\chapter{Bolometric Interferometry to Spectral Imaging}

\section{Synthesized Beam}

\section{Spectral Imaging}

\subfile{chapters/gps_calsource/gps_calsource.tex}

\chapter{Qubicsoft development}

\part{Map-Making}

\chapter{Frequency Map-Making}
% See FMM paper.

\chapter{Component Map-Making}
% See CMM paper.

\chapter{Realistic TOD Simulation}
% Chapter order to be reviewed.

\section{How to define Realistic TOD}

\section{Convolution approximation}

\section{TOD convergence problem: Pixelisation vs Number of Frequency}

\section{\texorpdfstring{Effect of $N_{\text{sub,TOD}} = N_{\text{sub,rec}}$}{Effect of Nsub\_TOD = Nsub\_rec}}

\section{External data addition effect (Planck)}

\section{Simulation hyperparameters optimisation: $N_{\text{sub}}$, $N_{\text{pointings}}$, $N_{\text{iter}}$, $N_{\text{loop}}$ (CMM), \dots}


\chapter{Neural-Network approximation of the Inverse Map-Making}
% Wait NNMM paper.

\part{Components separation and Cosmological parameters' estimation.}

\chapter{Atmospheric Mitigation}
% Wait Atm paper.

\chapter{QUBIC Forecasts}

\section{Studied Cases}

\subsection{CMB without foregrounds: QUBIC raw sensitivity}

\subsection{CMB + Dust (d0, d1, d6, \dots): Simple forecasts}

\subsection{CMB + Dust (d0, d1, d6) + Synchrotron (s0, s1): Realistic forecasts}

\section{Results}

\subsection{Compare QUBIC configurations}

\subsection{Dual bands without spectral imaging: ``Imager case''}

\subsection{Dual bands with spectral imaging: ``BI case''}

\subsection{Ultra-wide band: ``Extreme BI case''}

\subsection{Successive mono-band case: ``Conservative case''}

\section{Compare QUBIC's map-making algorithms}

\subsection{FMM: Cross-spectra Analysis}

\subsection{NN-FMM: Cross-spectra Analysis}

\subsection{CMM: Parametric Analysis}

\subsection{NN-CMM: Parametric Analysis}

\subsection{CMM: Blind Analysis}

\subsection{Addition of external data for component separation}

\subsection{Exploring exotic models}

% Bibliography
\addstarredpart{Bibliography}
\printbibliography

\end{document}
